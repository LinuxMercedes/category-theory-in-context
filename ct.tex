\documentclass{report}

% Commutative Diagrams
\usepackage{tikz}
\usetikzlibrary{cd}

% Must-have math packages
\usepackage{amsmath,amssymb}

% Theorem and proof formatting
\usepackage{amsthm}

% Reformat proof environment to end proof title with a : and a linebreak
\makeatletter
\renewenvironment{proof}[1][\proofname]{\par
	\pushQED{\qed}%
	\normalfont \topsep6\p@\@plus6\p@\relax
	\trivlist
\item\relax
	{\itshape
		#1\@addpunct{:}}\newline\indent
}{%
	\popQED\endtrivlist\@endpefalse
}
\makeatother

% -><- symbol for contradictions
\newcommand{\contradiction}{\ensuremath{\rightarrow\!\leftarrow}}

% Customizable numbering for enumerate
\usepackage{enumerate}

% Environment for writing problems.
% Problems are by default numbered by section, but can be explicitly numbered.
% Explicit numbering causes automatic numbering to start again at the explicit number
% (i.e., if you explicitly set the number of a problem to 6, the next automatically numbered
% problem will be 7).
\newcounter{problem}[section]
\usepackage{xifthen} % provides \isempty
\makeatletter
\newenvironment{problem}[1][]{\par
	\ifthenelse{\isempty{#1}}{%
		\stepcounter{problem}
	}{%
		\setcounter{problem}{#1}
	}
	\normalfont \topsep6\p@\@plus6\p@\relax
	\trivlist
	\item\relax
	{\bfseries
		Problem \roman{problem}\@addpunct{:}}\hspace\labelsep\ignorespaces
}{%
	\endtrivlist\@endpefalse
}
\makeatother

% Make a lemma environment, numbered sequentially
\newtheorem{lemma}{Lemma}

\newcommand{\category}[1]{\mathsf{#1}}
\newcommand{\inv}{^{-1}}

\title{Category Theory in Context \\ Solutions to Selected Exercises}
\author{Nathan Jarus}
\date{\today}
\begin{document}
\maketitle
\tableofcontents

\chapter{Categories, Functors, Natural Transformations}
\section{Abstract and concrete categories}

\begin{problem}
	\begin{enumerate}[(i)]
		\item Show that a morphism can have at most one inverse isomorphism.
		\item Consider a morphism $f : x \rightarrow y$.
			Show that if there exists a pair of morphisms $g,h:y \rightrightarrows x$ so that $gf = 1_x$ and $fh = 1_y$, then $g = h$ and $f$ is an isomorphism.
	\end{enumerate}
\end{problem}

\begin{proof}[Proof of (i)]
	Let $f:x \rightarrow y$ be a morphism and let $g,h:y \rightrightarrows x$ be morphisms such that
	$gf = hf = 1_x$ and $fg = fh = 1_y$.

	Then $g = g1_y = gfh = 1_x h = h$.
\end{proof}

\begin{proof}[Proof of (ii)]
	Let $f:x \rightarrow y$ be a morphism and let $g,h:y \rightrightarrows x$ be morphisms such that
	$gf = 1_x$ and $fh = 1_y$.

	Then $g = g1_y = gfh = 1_x h = h$.

	Therefore, $gf = 1_x$ and $fg = fh = 1_y$, so $f$ is an isomorphism.
\end{proof}

\begin{problem}
	Let $\category{C}$ be a category.
	Show that the collection of isomorphisms in $\category{C}$ defines a subcagegory, the \textbf{maximal groupoid} inside $\category{C}$.
\end{problem}

\begin{proof}
	Let $\category{G}$ be the maximal groupoid of $\category{C}$. We want to show that $\category{G}$ is a category.

	By definition, the objects of $\category{G}$ are the objects of $\category{C}$ and the morphisms of $\category{G}$ are the isomorphisms in $\category{C}$.

	For any identity morphism $1_x:x \rightarrow x$ in $\category{C}$, $1_x1_x = 1_x$, so $1_x$ is an isomorphism.
	Thus $\category{G}$ has all identity morphisms in $\category{C}$.

	For any morphisms $f:x \rightarrow y$ and $g:y \rightarrow z$ in $\category{G}$, there exist inverse morphisms $f\inv : y \rightarrow x$ and $g\inv : z \rightarrow y$ with $f\inv f = 1_x$, $ff\inv = 1_y$, $g\inv g = 1_y$, and $gg\inv = 1_z$.

	Thus, $g f f\inv g\inv = g 1_y g\inv = gg\inv = 1_z$ and $f\inv g\inv g f = f\inv f = 1_x$, so $fg$ is an isomorphism.
	Therefore, $\category{G}$ has composite morphisms.
\end{proof}

\begin{problem}
	For any category $\category{C}$ and any object $c \in \category{C}$, show that:

	\begin{enumerate}[(i)]
		\item There is a category $c/\category{C}$ whose objects are morphisms $f:c \rightarrow x$ with domain $c$
			and in which a morphism from $f:c \rightarrow x$ to $g:c \rightarrow y$ is a map $h:x \rightarrow y$ between the codomains
			so that the triangle

			\begin{tikzcd}
				& c \arrow[ld, "f"'] \arrow[rd, "g"] &  \\
				x \arrow[rr, "h"'] &  & y
			\end{tikzcd}

			commutes, i.e., so that $g = hf$.

		\item There is a category $\category{C}/c$ whose objects are morphisms $f:x \rightarrow c$ with codomain $c$
			and in which a morphism from $f:x \rightarrow c$ to $g:y \rightarrow c$ is a map $h:x \rightarrow y$ between the domains
			so that the triangle

			\begin{tikzcd}
				x \arrow[rr,"h"] \arrow[rd, "f"'] &  & y \arrow[ld, "g"] \\
				& c &
			\end{tikzcd}

			commutes, i.e., so that $f = gh$.
	\end{enumerate}
\end{problem}

% Identities are the same as in C: f:c->x has identity 1_x; h1_xf = hf = g
% s:f->g t:g->h  => exists u:f->h where u:x->z in C

\begin{proof}[Proof of (i)]
To avoid confusion, we will write objects and morphisms in $c/\category{C}$ in a bold font to distinguish them from objects and morphisms in $\category{C}$.

The objects in $c/\category{C}$ are morphisms in $\category{C}$ where $\mathbf{f} \in c/\category{C}$ if $f : c \rightarrow x$ is a morphism in $\category{C}$.
A morphism $\mathbf{s}: \mathbf{f} \rightarrow \mathbf{g}$ exists in $c/\category{C}$
if morphisms $f: c \rightarrow x$, $g:c \rightarrow y$ and $s:x \rightarrow y$ exist in $\category{C}$ such that $g = sf$.

The identity morphism $\mathbf{1_f}$ for the object $\mathbf{f}: c \rightarrow x$ in $c/\category{C}$ is the identity morphism $1_x$ in $\category{C}$.
For any $\mathbf{s}: (\mathbf{f}: c \rightarrow x) \rightarrow (\mathbf{g}:c \rightarrow y)$, we have
$\mathbf{s}\mathbf{1_f} = s 1_x = \mathbf{s}$ and $\mathbf{1_g}\mathbf{s} = 1_y s = \mathbf{s}$.

If we have morphisms $\mathbf{s} : (\mathbf{f}: c \rightarrow x) \rightarrow (\mathbf{g}:c \rightarrow y)$ and
$\mathbf{t} : (\mathbf{g}: c \rightarrow y) \rightarrow (\mathbf{h}:c \rightarrow z)$ in $c/\category{C}$, there exist morphisms
$s : x \rightarrow y$ and $t : y \rightarrow z$ in $\category{C}$ where $g = sf$ and $h = tg$.
Thus, $ts: x \rightarrow z$ is in $\category{C}$ and $tsf = tg = h$.
Therefore, $\mathbf{ts}: \mathbf{f} \rightarrow \mathbf{h}$ is a morphism in $c/\category{C}$.

In other words, the following diagram commutes:

\begin{tikzcd}
 & c \arrow[ld, "f"'] \arrow[d, "g"] \arrow[rd, "h"] &  \\
x \arrow[r, "s"] \arrow[rr, "ts"', dashed, bend right] & y \arrow[r, "t"] & z
\end{tikzcd}
\end{proof}

\begin{proof}[Proof of (ii)]
The objects in $\category{C}/c$ are morphisms in $\category{C}$ where $\mathbf{f} \in \category{C}/c$ if $f : x \rightarrow c$ is a morphism in $\category{C}$.
A morphism $\mathbf{s}: \mathbf{f} \rightarrow \mathbf{g}$ exists in $\category{C}/c$
if morphisms $f: x \rightarrow c$, $g:y \rightarrow c$ and $s:x \rightarrow y$ exist in $\category{C}$ such that $f = gs$.

The identity morphism $\mathbf{1_f}$ for the object $\mathbf{f}: x \rightarrow c$ in $\category{C}/c$ is the identity morphism $1_x$ in $\category{C}$.
For any $\mathbf{s}: (\mathbf{f}: x \rightarrow c) \rightarrow (\mathbf{g}:c \rightarrow y)$, we have
$\mathbf{s}\mathbf{1_f} = s 1_x = \mathbf{s}$ and $\mathbf{1_g}\mathbf{s} = 1_y s = \mathbf{s}$.

If we have morphisms $\mathbf{s} : (\mathbf{f}: x \rightarrow c) \rightarrow (\mathbf{g}:y \rightarrow c)$ and
$\mathbf{t} : (\mathbf{g}: y \rightarrow c) \rightarrow (\mathbf{h}:z \rightarrow c)$ in $\category{C}/c$, there exist morphisms
$s : x \rightarrow y$ and $t : y \rightarrow z$ in $\category{C}$ where $f = gs$ and $g = ht$.
Thus, $ts: x \rightarrow z$ is in $\category{C}$ and $hts = gs = f$.
Therefore, $\mathbf{ts}: \mathbf{f} \rightarrow \mathbf{h}$ is a morphism in $\category{C}/c$.

In other words, the following diagram commutes:

\begin{tikzcd}
x \arrow[rd, "f"'] \arrow[r, "s"] \arrow[rr, "ts", dashed, bend left] & y \arrow[d, "g"] \arrow[r, "t"] & z \arrow[ld, "h"] \\
 & c &
\end{tikzcd}
\end{proof}


\end{document}